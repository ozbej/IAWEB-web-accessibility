%----------------------------------------------------------------
%
%  File    :  survey-intro.tex
%
%  Author  :  Keith Andrews, IICM, TU Graz, Austria
% 
%  Created :  27 May 1993
% 
%  Changed :  16 Nov 2010
% 
%----------------------------------------------------------------


\chapter{Introduction}

\label{chap:Intro}



An academic survey paper presents a survey or overview of the state of
the art in a particular field. Every chapter and every section should
have some introductory text at the beginning, like this text. Never
jump straight in to the first secion or subsection without one or more
paragraphs of introductory text.






\section{Not a Series of Summaries}

A survey is \emph{not} simply a series of summaries of papers.
If I have given you say 8 papers to start you off, what you should
\emph{not} do is: divide up the papers (read two each) and produce a
series of 8 unconnected paper summaries.




\section{Read All the Papers and Research Some More}

Each of you should read \emph{all} the papers and resources: both
those I gave you and those you found yourselves.
%
Make sure you search for more papers and resources yourselves. Not
just a Google search. Search the ACM \parencite{ACM-DL} and IEEE
\parencite{IEEE-DL} digital libraries. You may want to use Mendeley to
collect your resources or maybe maintain a (shared) \fname{.bib} file
within a Git repository.

Include a list of \emph{all} the relevant papers and resources you
have found and mark those you have chosen to focus on. Make sure
\emph{all} the papers and resources you found or were given appear in
the bibliography.




\section{Dividing up the Field}

The hardest part of any survey is dividing up the field.  Look for
common concepts and threads in the papers and resources. Do they
report similar or dissimilar results? Does one paper or resource
support or contradict another?

Once you have all read all the papers: you need to construct a small
hierarchy (taxonomy) to classify the concepts appearing in the papers
and resources. Structure your survey into chapters and sections based
on your taxonomy.






\section{Composing a Title and Abstract}

One useful strategy for composing a good title and abstract involves
brainstorming for a list of keywords. Start by writing down a list of
all the words and phrases describing important topics covered in the
thesis and which potential interested readers might use as search
terms to find the thesis. Then construct a title containing the most
important of these keywords. Finally, compose the abstract and make
sure most of the rest of the keywords are contained somewhere in the
abstract. Search engines and library systems will usually index the
title and the abstract, so anyone searching for any of the keywords
should now be able to find the thesis. When the thesis is approaching
completion, revisit the title and abstract, an extra extra keywords
and make any necessary adaptations.




\section{Double-Sided Printing}

Create and print your survey in colour and for two-sided (duplex)
printing. Modern laser printers can easily handle printing out in
colour and double-sided. A survey paper printed one-sided will be
(unnecessarily) twice as thick and twice as heavy.

Sections, including the bibliography and any appendices, should
usually (as far as possible) start on a new right-hand (odd-numbered)
page. This is what the \lstinline!\cleardoublepage! command does.






\section{Single Children}

As in real life, a single child is not a good idea. A chapter with
only one section makes no sense. A section with only one subsection
makes no sense. A subsection with only one subsubsection makes no
sense either. If a structural unit has subunits, then there should
always be at least two subunits.







\section{Make Captions Carry the Story Too}

Some readers like to scan through your work from figure to figure,
gaining an impression of what it is about by reading the captions.
Support these readers by:
\begin{itemize}
\item \emph{Writing self-contained captions}: the caption
should describe the figure or table as completely as possible, without
assuming knowledge of material in the running text.

\item \emph{Writing longish captions}: it is fine for
captions to contain two or three sentences.

\item \emph{Stringing captions together}: Reading successive captions
should also tell an abridged version of the entire story.
\end{itemize}




\section{Avoid Orphan Floats}

Every floating element (figure, table, or listing) which appears in
the thesis and is given its own number such as Figure~3.1, Table~4.1,
or Listing~5.1 \emph{must} be discussed and referenced somewhere in the
running text. An orphan float is a float which appears and has a
number, but is never referenced in the flowing text.



\chapter{Browser Extensions for Accessibility Auditing}

\label{chap:Intro}

Using browser extensions is one way to audit the accessibility of websites.
This paper evaluates five different extensions for accessibility auditing: axe DevTools, Accessibility Insights for Web, Google Lighthouse, Siteimprove and WAVE.
Three of the tools we have selected for this paper use the \emph{axe-core} library, which is an open-source accessibility engine for automated web UI testing.
As a result, many of the tools will give similar results and the main difference between the extensions is how the list of accessibility issues is presented and what additional features are offered.
The library allows accessibility auditing to Web Content Accessibility Guidelines (WCAG) 2.0 and 2.1 on the levels A and AA.

Most of the browser extensions evaluated in this paper are completely free.
Only one of the tools evaluated in this paper offers a paid version of the extension with additional features.

\begin{table}[tp]
\tablestretch
\rowcolors{2}{}{tablerowcolour}
\centering
\begin{tabularx}{\linewidth}
{>{\kern-\tabcolsep}lllXX<{\kern-\tabcolsep}}
\toprule
\textbf{Extension} & \textbf{Browser} & \textbf{Licence} & \textbf{Downloads\footnotemark}
\\
\midrule
axe DevTools & Chrome, Edge & Free, commercial & 210000 \\
%
Accessibility Insights for Web & Chrome, Edge & Free, open-source & 120000 \\
%
Google Lighthouse & Chrome (built-in) & Free, open-source & 900000\footnotemark \\
%
Siteimprove Accessibility Checker & Chrome, Firefox, Edge, Opera & Free & 30000 \\
%
WAVE Evaluation Tool & Chrome, Firefox, Edge & Free & 520000 \\
\bottomrule
\end{tabularx}
\caption[Browser Extensions Information]
{
Browser extensions information \\
% TODO: Figure out how to use \footnotetext here
\textsuperscript{1}\text{Data collected from Chrome Web Store, Opera Addons, Firefox Add-ons and Microsoft Edge Add-ons.} \\
\textsuperscript{2}\text{Google Lighthouse is built-in into Google Chrome, so the actual number of users may be significantly higher.}
}
\label{tab:browser-extensions-info}
\footnotetext{test}
\end{table}

\section{axe DevTools}
Axe DevTools is a tool for auditing accessibility developed by Deque Systems Inc.
The browser extension is based on the \emph{axe-core} underlying library for auditing the accessibility, but also offers some additional stricter audit rules compared to the base \emph{axe-core} library, allowing validation according to WCAG 2.1 AAA.
The extension is available for Chrome and Edge.
The Deque website contains a link to the axe DevTools extension for Firefox, but at the time of writing this paper it does not seem to work.

There are three different plans available, Free, Pro and Enterprise.
The free version has a limited set of features and mostly consists of only the automated testing of accessibility.
The Pro version costs \$40 per month and offers additional features, such as guided manual tests, partial accessibility testing and exporting of accessibility issues.
Lastly, the Enterprise version of axe DevTools offers even more features, for example CI/CD integration and custom rules.


\section{Accessibility Insights for Web}
Accessibility Insights for Web is a browser extension for accessibility auditing using Chrome or Edge.
The extension offers two different audit methods, automated checking, known as FastPass, and manual assessment.
The FastPass automated check provides accessibility auditing using the the axe-core library.
The accessibility issues are then visible in a list and can be exported as an HTML report or directly to GitHub Issues or Azure Boards for easy integration in the development process.
Another feature available in the extension is the visualization of accessibility issues directly on the web page.
This allows scrolling the web page to see where accessibility issues occur .
The manual assessment in the browser extension consists of extensive step-by-step instructions for auditing the accessibility manually.
Of the browser extensions evaluated in this paper, Accessibility Insights for Web is the only free tool to offer support for manual testing.


\section{Google Lighthouse}
Google Lighthouse is a the most popular tool for auditing accessibility, which is included in Google Chrome.
Lighthouse, like the previous browser extensions, is also based on the \emph{axe-core} library.
For auditing accessibility, Lighthouse offers a relatively basic set of features, and relies heavily on integration with Chrome DevTools and external references for accessibility issue descriptions.
Accessibility issues are presented as a list of what accessibility checks have passed and which ones have failed.
The user can then see descriptions about the accessibility issues and see where in the HTML code they occur.
Reports of the accessibility audit can be saved as JSON.

\section{Siteimprove Accessibility Checker}
Siteimprove Accessibility Checker is the least used accessibility audit browser extension out of the extensions in this survey.
The extension presents the accessibility issues, where you can expand each issue to read the description of it and highlight the element on the web page.
Some features that the extension offers, that the other extensions do not offer is simulation of color blindness and excellent filtering of accessibility issues.
The list of issues can be filtered by difficulty, role, WCAG level and HTML element type.
However, one feature that is not offered is the possibility of exporting the audit report.

\section{WAVE Evaluation Tool}
The WAVE evaluation tool is another extension for auditing accessibility, supporting Chrome, Firefox and Edge.
The extension functions somewhat differently than the other extensions, as the tool is embedded directly on the web page.
It then inserts icons for elements that fail or succeed in the accessibility checks.
The user can then scroll the web page and click the icons to learn more about the accessibility issue.

\section{Showcase videos}
For each of the browser extension, we have recorded videos to showcase the extension and their features.
Using each extension, we performed an accessibility audit of http://www.gov.uk/ http://www.mirror.co.uk/.
The showcase videos for axe DevTools \parencite{axe_ext_vid}, Accessibility Insights for Web \parencite{aiweb_ext_vid}, Google Lighthouse \parencite{lighthouse_ext_vid}, Siteimprove Accessibility Checker \parencite{siteimprove_ext_vid}, WAVE Evaluation tool are available online \parencite{wave_ext_vid}.

\section{Browser Extensions Conclusion}
TODO: add conclusion

